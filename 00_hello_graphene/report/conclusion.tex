\begin{center}
    ЗАКЛЮЧЕНИЕ
\end{center}
\addcontentsline{toc}{chapter}{Заключение}

В данной работе были рассмотрены 3 пакета квантовохимических вычислений, с их помощью была получена зонная структура листа графена. Два из них, ABINIT и QUANTUM ESPRESSO, продемонстрировали схожие с экспериментальными данными результаты, в то время как результаты CRYSTAL отличались большей погрешностью. Учитывая этот факт, а так же то, что первые два из трёх рассмотренных пакетов бесплатные, для проведения дальнейших расчётов стоит отдать предпочтение одному из них, используя другой для проверки результатов.
