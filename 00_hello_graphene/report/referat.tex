\begin{center}
    РЕФЕРАТ
\end{center}

В работе сделан краткий обзор основ теории функционала плотности, на которой основаны современные пакеты квантовой химии. Рассмотрены три таких пакета: ABINIT, QUANTUM ESPRESSO и CRYSTAL. С их помощью произведён расчёт зонной структуры бесконечного листа графена. Проведено сравнение полученных результатов.

\vspace{1cm}
\noindent Ключевые слова: ТФП, графен, зонная структура, конус Дирака, ABINIT, CRYSTAL, QUANTUM ESPRESSO

\vspace{2cm}
\begin{center}
    ABSTRACT
\end{center}

In this paper brief overview of the basics of density functional theory, which  modern quantum chemistry packages is based on, is given. Consider three packages: ABINIT, QUANTUM ESPRESSO and CRYSTAL. With their help, the calculation of the band structure of an infinite graphene sheet were promoted. The results are compared.

\vspace{1cm}
\noindent Keywords: DFT, graphene, band structure, Dirac cone, ABINIT, CRYSTAL, QUANTUM ESPRESSO