\begin{center}
    ПРИЛОЖЕНИЕ А\\
    Файл конфигурации для ABINIT
\end{center}
\addcontentsline{toc}{chapter}{Приложение А Файл конфигурации для ABINIT}
\begin{Verbatim}[fontfamily=courier,fontsize=\footnotesize]
    ndtset 2                      2 набора данных

    kptopt1 1                     автогенерация точек в обратной решётке
    chksymbreak 0                 игнорировать нарушения симметрии
    ngkpt1 10 10 1                размеры сетки
    prtden1  1                    сохранить полученную электронную плотность
    toldfe1  1.0d-8               точность

    iscf2    -2                   использовать сохранённые данные
    getden2  -1                   использовать плотность, полученную в 1
    nband2   22                   число зон
    kptopt2  -2                   число линий, вдоль которых считаются зоны
    ndivk2   20 19                число точек на каждой линии
    kptbounds2                    концы линий
    1/2   0 0                     M
      0   0 0                     Г
    1/3 1/3 0                     К
    enunit2   1                   выводить энергию в эВ
    tolwfr2  1.0d-12              точность

    acell 2.46 2.46 30 angstrom   размеры примитивной ячейки и векторы
    rprim
           1          0 0
        -1/2 sqrt(0.75) 0
           0          0 1

    ntypat 1                      число различных типов атомов
    znucl 6                       заряд ядра
    natom 2                       число атомов в примитивной ячейке
    typat 1 1                     типы этих атомов
    xred                          их положение в ячейке
      0   0 0
    2/3 1/3 0
    nstep 50                      максимальное число итераций
    ecut 40                       максимальная энергия плоской волны
\end{Verbatim}
\newpage
\begin{center}
    ПРИЛОЖЕНИЕ Б\\
    Файлы конфигурации для QUANTUM ESPRESSO
\end{center}
\addcontentsline{toc}{chapter}{Приложение Б Файлы конфигурации для QUANTUM ESPRESSO}
Расчёт электронной плотности
\begin{Verbatim}[fontsize=\footnotesize]
&CONTROL
    calculation = 'scf',          тип вычислений: вычисления электронной плотности
    prefix='graphene',
    pseudo_dir = "./pseudo",       путь к псевдопотенциалам
    outdir = "./temp",
/
&SYSTEM
    ibrav=0,                      тип решётки Браве: задаётся вручную
    celldm(1)=4.6595,             размер примитивной ячейки
    nat=2,                        число атомов в ней
    ntyp=1,                       число разновидностей атомов
    ecutwfc=70.0,                 максимальная энергия волновой функции
/
&ELECTRONS
/
ATOMIC_SPECIES
    C 12 C.pbe-rrkjus.UPF         выбор псевдопотенциала
CELL_PARAMETERS                   параметры примитивной ячейки
    1.0 0.0 0.0
    -0.5 0.8660254 0.0
    0.0 0.0 12.0
ATOMIC_POSITIONS (crystal)        расположение атомов
   C        0.000000000   0.000000000   0.000000000
   C        0.666666666   0.333333333   0.000000000
K_POINTS {Automatic}              размеры рассчётной сетки в обратной решётке
   16 16 1 0 0 0
\end{Verbatim}
\vspace{1cm}
Расчёт зонной структуры
\begin{Verbatim}[fontsize=\footnotesize]
&CONTROL
    calculation = 'bands',    тип вычислений: зонная структура
    prefix='graphene',
    pseudo_dir = "./pseudo",
    outdir = "./temp",
/
&SYSTEM
    ibrav=0,
    celldm(1)=4.6595,
    nat=2,
    ntyp=1,
    ecutwfc=70.0,
    nbnd = 22,
/
&ELECTRONS
/
ATOMIC_SPECIES
    C 12 C.pbe-rrkjus.UPF
CELL_PARAMETERS
    1.0 0.0 0.0
    -0.5 0.8660254 0.0
    0.0 0.0 30.0
ATOMIC_POSITIONS (crystal)
   C        0   0   0
   C        0.6666666 0.3333333 0
K_POINTS crystal_b            способ задания точек: в базисе векторов обратной решётки
   3                          число точек   
   0.5 0 0 19                 M
   0 0 0 20                   Г
   0.33333 0.33333 0 10       K
\end{Verbatim}
\newpage
\begin{center}
    ПРИЛОЖЕНИЕ В\\
    Файлы конфигурации для CRYSTAL
\end{center}
\addcontentsline{toc}{chapter}{Приложение В Файлы конфигурации для CRYSTAL}
Расчёт электронной плотности
\begin{Verbatim}[fontsize=\footnotesize]
Graphene
SLAB                                     2D кристалл
77                                       тип решётки
2.46                                     размер ячейки
1                                        число несимметричных атомов в ячейке
6  0.66666666666 0.33333333333 0.        их положения
END
6  2                                     базис
1 0 3  2.  0.
1 1 3  4.  0.
99 0
END
DFT                                      использовать при расчётах DFT
EXCHANGE                                 вид обменного потенциала
PBE
CORRELAT                                 вид корреляционного потенциала
PBE
END
SHRINK
8 8                                      размер сетки в обратной решётке
END
\end{Verbatim}
\vspace{1cm}
Расчёт зонной структуры
\begin{Verbatim}[fontsize=\footnotesize]
NEWK
16 16                                    размер сетки в обратной решётке
1 0
BAND
bands of graphene
2 6 100 3 10 1 0
3 0 0 0 0 0                              M - Г
0 0 0 2 2 0                              Г - K
END
\end{Verbatim}
\newpage
\begin{center}
    ПРИЛОЖЕНИЕ Г\\
    Скрипт для обработки и визуализации данных
\end{center}
\addcontentsline{toc}{chapter}{Приложение Г Скрипт для обработки и визуализации данных}
\begin{Verbatim}[fontsize=\footnotesize]
ab = Import["ab.csv"];
qe = Import["qe.csv"];
cr = Import["cr.csv"];

(* перестановка элементов в массиве data для минимизации |data - p| *)
perm[data_, p_] := Module[{a},
    a = data;
    Do[
        Do[
            If[ Abs[a[[j]] - p[[i]]] < Abs[a[[i]] - p[[i]]],
                {a[[i]], a[[j]]}={a[[j]], a[[i]]}],
            {j, i, Length[p]}],
        {i, Length[p]}];
    a]

(* разбирает зоны по производной *)
bands[data_] := Module[{result},
    result = data[[;;2]];
    Do[
        AppendTo[result, perm[data[[i]], 2 result[[-1]] - result[[-2]]]],
        {i, 3, Length[data]}];
    Transpose[result]]

result = Join[bands[ab],bands[qe],bands[cr]];
(* вывод графиков *)
Export["bands.png",
        ListLinePlot[result, PlotRange -> All]];
\end{Verbatim}