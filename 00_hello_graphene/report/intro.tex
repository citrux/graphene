\begin{center}
    ВВЕДЕНИЕ
\end{center}
\addcontentsline{toc}{chapter}{Введение}

Для изучения особенностей зонной структуры твёрдых тел в современной теоретической физике используются различные квантовохимические пакеты. Многие из них основаны на методах DFT --- теории функционала плотности (ТФП), разработанной в 60-x годах прошлого века. Для проведения расчётов в моей магистерской работе следует ознакомиться с некоторыми из таких пакетов, научиться задавать в них конфигурацию исследуемой системы и извлекать необходимые результаты, полученные в результате работы. 

В этой работе рассмотрены 3 пакета: свободные ABINIT и QUANTUM ESPRESSO, а также проприетарный CRYSTAL (демонстрационная версия).
