\section{QUANTUM ESPRESSO}
QUANTUM ESPRESSO реализует различные методы и алгоритмы, направленные на химически реалистичное моделирование материалов, начиная с наноразмерных, на основе решения задачи теории функционала плотности (DFT) с использованием базиса плоских волн и псевдопотенциалов, чтобы представить электрон-ионные взаимодействия.

Программа построена на использовании периодических граничных условий, что позволяет просто рассчитывать бесконечные кристаллические системы. Конечные системы рассчитываются с помощью сверхячеек. QUANTUM ESPRESSO таким образом, может быть использован для любой кристаллической структуры или сверхячейки как металлов, так и диэлектриков. Атомные ядра могут быть описаны различными наборами волн. Имеется множество различных обменно-корреляционных функционалов локальной плотности (LDA) или обобщенного градиентного приближений (GGA), а также такие функционалы, как U-поправки Хаббарда, несколько мета-GGA и гибридные функционалы.

Ниже представлен неполный список расчетов, которые могут быть выполнены при помощи программы\cite{qe}:
\begin{itemize}
    \item расчет Кона-Шэма (KS) орбиталей и энергий для изолированных или периодических систем, и их энергий основного состояния;
    \item полные структурные оптимизации микроскопических (атомные координаты) и макроскопических (ячейки) степеней свободы, используя силы и напряжения Гельмана-Фейнмана;
    \item расчёт основного состояния магнитных или спин-поляризованных систем, в том числе спин-орбитального взаимодействия и неколлинеарного магнетизма;
    \item молекулярная динамика, используя либо лагранжиан Кара-Парринелло, либо силы Гельмана-Фейнмана, рассчитанные на поверхности Борна-Оппенгеймера;
    \item теория возмущений функционала плотности (DFPT);
    \item получение максимально локализованных функций Ванье и связанных с ними величин;
    \item расчет параметров ядерного магнитного резонанса (ЯМР) и электронного парамагнитного резонанса (ЭПР);
    \item расчет К-края спектров рентгеновского поглощения.
\end{itemize}
