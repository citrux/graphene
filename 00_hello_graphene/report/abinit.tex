\chapter{Квантовохимические пакеты}
\section{ABINIT}
ABINIT основывается на теории функционала плотности (ТФП). Для функций отклика, таких как вибрации, диэлектрические или пьезоэлектрические свойства, реализована специализированная версия ТФП, теория возмущений функционала плотности, обеспечивающая эффективность и простоту использования. Основные возможности ABINIT реализуются с помощью этих двух методов ТФП. Тем не менее, для расчёта электронных свойств, в частности, ширины запрещенной зоны, методы теории функционала плотности, как известно, ненадежны. В ABINIT, такие электронные свойства могут быть вычислены точно благодаря реализации многочастичной теории возмущений, так называемого <<GW приближения>>.

В программе реализованы различные приближения функционала обменно-корреляционной энергии, в том числе различных видов приближении локальной плотности и обобщенного градиентного приближения. С помощью этих приближений, длины связей и углы, как правило, предсказываются с точностью в пределах нескольких процентов.

ABINIT основан на разложении электронных волновых функций на плоские волны, с периодическим представлением системы в ящике при периодических граничных условиях. Это представление особенно подходит для исследования кристаллов: ящик принимается в качестве примитивной элементарной ячейки. Если принимать непримитивную ячейку (или сверх-ячейку), то можно изучать системы, в которых трансляционная симметрия отсутствует. Чтобы дать некоторое представление о размерах разрешенных систем, отметим, что для тестирования ABINIT были использованы системы, содержащие до 250 атомов, в то время как в большинстве ТФП расчётов ABINIT размер ячейки не превышает 50 атомов, и даже, в случае более требовательного GW исследования, число атомов не достигает и десятка.

Псевдопотенциалы позволяют не рассматривать внутренние электроны и сосредоточить внимание на формировании связей и свойствах валентных электронов. ABINIT имеет обширную библиотеку псевдопотенциалов для всей таблицы Менделеева.

С помощью ABINIT достаточно просто изучать как металлы, так и диэлектрики: сетки волновых векторов, необходимые для расчёта вклада каждого электрона в зоне Бриллюэна создаются автоматически. Для металлов, различные схемы размытия позволяют уменьшить количество таких волновых векторов.

Основным результатом ТФП является электронная плотность. В ABINIT, из-за использования псевдопотенциалов рассматриваются плотности валентных электронов. 

Как уже упоминалось, использование ТФП для расчёта зонной структуры находится под вопросом, и, следовательно, должно быть ограничено качественным анализом. Это особенно важно при исследовании вещества на металлические или диэлектрические свойства или количественном расчёте ширины запрещенной зоны.

Методология GW, наоборот, как правило, позволяет достичь очень хорошего описания зонной структуры и запрещенных зон, с погрешностью менее 0,2 эВ по отношению к экспериментальным данным. Расчет GW выполняется для заданной геометрии (оптимизированной или экспериментальной), после выполнения расчета ТФП (необходимы электронные волновые функции).\cite{abinit}
