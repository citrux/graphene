\section{CRYSTAL}
Пакет CRYSTAL выполняет первопринципные расчеты энергии основного состояния, градиента энергии, электронной волновой функции и свойств периодических систем. Могут быть использованы гамильтонианы Хартри-Фока или Кона-Шэма. На равных основаниях рассматриваются периодические системы в 0 (молекулы, 0D), 1 (полимеры, 1D), 2 (листы, 2D) и 3-х измерениях (кристаллы, 3D). В каждом случае в качестве приближения используется разложение волновых функций одной частицы (CO) в виде линейной комбинации функций Блоха (BF) определённых с помощью локальных функций (АО).

Локальные функции, в свою очередь, представляют из себя линейные комбинации функций гауссового типа (GTF), показатели и коэффициенты которых определяются входными данными. Могут быть использованы функции симметрии s, p, d и f. Также доступны sp-оболочки (s- и р-оболочки с общим набором показателей). Использование sp оболочек может привести к значительной экономии процессорного времени.

Программа может автоматически обрабатывать пространственные симметрии: 230 пространственных групп, 80 групп слоев, 99 одномерных группы, 45 точечные группы. Точечные симметрии, совместимые с трансляционной симметрией, предназначены для молекул.

Программа позволяет создавать специальные 1D (нанотрубки) и 0D (фуллерены) структуры из 2D структур, а также генерировать листы (2D системы) или кластеры (система 0D) из 3D-кристаллической структуры, упругие деформации решетки, создавать сверх-ячейки с дефектом и разнообразно редактировать структуры.\cite{crystal}