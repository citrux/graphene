\chapter{Теория функционала плотности}
Одна из самых главных проблем теоретической химии твердого тела состоит в разработке соответствующих методов для решения многоэлектронного уравнения Шредингера. Метод Хартри-Фока обеспечивает приближенное решение для задачи N частиц, но не подходит для больших систем. Наиболее часто используемыми методами являются методы на основе теории функционала плотности (ТФП).

ТФП, cформулированная Хоэнбергом и Коном (HK) и разработанная Коном и Шэмом (KS), основана на том, что энергия основного состояния системы является фнкционалом электронной плотности \(\rho(\vec{r})\). Волновая функция системы из \(N\) электронов содержит \(3N\) координат, по три для каждого электрона. Электронная плотность \(\rho(\vec{r})\) является квадратом волновой функции, интегрированным по координатам \(N-1\) электронов: \(\rho(\vec{r})\) зависит только от трех координат, независимо от числа электронов. Это означает, что, в то время как сложность волновой функции возрастает с увеличением числа электронов, плотность электронов имеет одинаковое число переменных, независимо от размера системы.

ТФП основана на двух теоремах. Первая теорема HK утверждает все электронные свойства системы в основном состоянии определяются электронной плотностью основного состояния.

Вторая теорема HK связывает основное состояние электронной плотности \(\rho_0(\vec{r})\) системы с ее полной энергией:
\begin{equation}
    E[\rho(\vec{r})] \ge E[\rho_0(\vec{r})] \equiv E.
\end{equation}
Полная энергия:
\begin{equation}
    E[\rho(\vec{r})] = T[\rho(\vec{r})] + V_{ee}[\rho(\vec{r})]+
    \int\limits_D \rho(\vec{r}) v(\vec{r}) d\vec{r},
    \label{eq:36}
\end{equation}
где \(T\) кинетическая энергия электронов, \(V_{ee}\) -- электрон-электронный вклад и \(v(\vec{r}\) является потенциалом, действующим на электроны, включающим в себя ядерно-электронного взаимодействия и внешний потенциал, действующей на систему. Так как выражения для \( T \) и \(V_{ee}\) точно известны, основное состояние может быть определено путем непосредственного минимизации функционала \eqref{36}.

Метод Кона-Шема состоит в принятии в качестве отсчёта невзаимодействующей электронной системы, для которой кинетическая энергия может быть вычислена. Электрон-электронные взаимодействия, следовательно, могут быть рассчитаны в классическом  приближении усредненного поля, в то время как все неклассические вклады (корреляции) в \(V_ee\) и вклад невзаимодействующих электронов в кинетическую энергию включены в корреляционно-обменный член:
\begin{equation}
    E_{xc}[\rho(\vec{r})] = V_{ee} - \frac{1}{2}\int\limits_{D}\frac{\rho(\vec{r})\rho(\vec{r}')}{|\vec{r} - \vec{r}'|}.
    \label{eq:37}
\end{equation}

\( \rho(\vec{r}) \)  может быть выражено через одноэлектронные функции \( \psi_i \):
\begin{equation}
    \rho(\vec{r}) = \sum_{i=1}^N |\psi_i|^2,
\end{equation}
где \(\psi_i\) являются решениями \( N \) одноэлектронных уравнений Кона-Шэма
\begin{gather}
    \hat{H}\psi_i = \eps_i\psi_i,\quad i=1,\ldots,N\\
    \hat{H} = \hat{T} + \hat{Z} + \hat{C} + \hat{E}_{xc}.
\end{gather}
Таким образом, в отличие от метода Хартри-Фока, в ТФП учитывается корреляционная энергия. Функционал \(E_{xc}\) определяется следующим образом:
\begin{equation}
    E_{xc} = \pder{E_{xc}[\rho(\vec{r})]}{\rho(\vec{r})}
\end{equation}
Если обменно-корреляционный член известен, уравнения KS обеспечивают точные решения. Для того, чтобы применить метод ТФП, однако, должен быть определен приблизительный вид \(Е_{хс}\): различия между ТФП методами заключаются в выборе его функционального вида. Простейшим решением является приближение локальной плотности (LDA):
\begin{equation}
    E_{xc}^{LDA}[\rho(\vec{r})] = \int\limits_{D}\eps_{xc}(\rho(\vec{r}))\rho(\vec{r}) d\vec{r},
\end{equation}
где \(\eps_{xc}(\rho(\vec{r}))\) является обменно-корреляционной энергии для частицы в однородном электронно-взаимодействующий газ с плотностью \(\rho(\vec{r})\).

Более точное приближение -- обобщённое градиентное приближение (GGA)
\begin{equation}
    E_{xc}^{GGA}[\rho(\vec{r})] = \int\limits_{D}f(\rho(\vec{r}), \nabla\rho(\vec{r})) d\vec{r},
\end{equation}
где \( f(\rho(\vec{r}), \nabla\rho(\vec{r})) \) является функцией как электронной плотности и её градиента. GGA позволяет лучше описывать неоднородные системы.

Наиболее часто используются комбинации функционалов с привлечением точного результата, полученного методом Хартри-- Фока, так как они, как правило, обеспечивают лучшую точность в оптимизации геометрии и энергетических расчетах. Они известны как гибридные методы.