\begin{center}
    Приложение А\\
    Файл конфигурации для ABINIT
\end{center}
\addcontentsline{toc}{chapter}{Приложение А Файл конфигурации для ABINIT}
\begin{Verbatim}[fontsize=\footnotesize]
    ndtset 2

    kptopt1 1
    chksymbreak 0
    ngkpt1 10 10 1
    prtden1  1
    toldfe1  1.0d-8

    iscf2    -2
    getden2  -1
    nband2   22
    kptopt2  -2
    ndivk2   20 19
    kptbounds2
    1/2   0 0
      0   0 0
    1/3 1/3 0
    enunit2   1
    tolwfr2  1.0d-12

    acell 2.46 2.46 30 angstrom
    rprim
           1          0 0
        -1/2 sqrt(0.75) 0
           0          0 1

    ntypat 1
    znucl 6
    natom 2
    typat 1 1
    xred
      0   0 0
    2/3 1/3 0
    nstep 50
    ecut 40
\end{Verbatim}
\newpage
\begin{center}
    Приложение Б\\
    Файлы конфигурации для QUANTUM ESPRESSO
\end{center}
\addcontentsline{toc}{chapter}{Приложение Б Файлы конфигурации для QUANTUM ESPRESSO}
Расчёт электронной плотности
\begin{Verbatim}[fontsize=\footnotesize]
&CONTROL
    calculation = 'scf',
    prefix='graphene',
    pseudo_dir = "./pseudo",
    outdir = "./temp",
/
&SYSTEM
    ibrav=0,
    celldm(1)=4.6595,
    nat=2,
    ntyp=1,
    ecutwfc=70.0,
/
&ELECTRONS
/
ATOMIC_SPECIES
    C 12 C.pbe-rrkjus.UPF
CELL_PARAMETERS
    1.0 0.0 0.0
    -0.5 0.8660254 0.0
    0.0 0.0 30.0
ATOMIC_POSITIONS (crystal)
   C        0.000000000   0.000000000   0.000000000
   C        0.666666666   0.333333333   0.000000000
K_POINTS {Automatic}
   16 16 1 0 0 0
\end{Verbatim}
\vspace{1cm}
Расчёт зонной структуры
\begin{Verbatim}[fontsize=\footnotesize]
&CONTROL
    calculation = 'bands',
    prefix='graphene',
    pseudo_dir = "./pseudo",
    outdir = "./temp",
/
&SYSTEM
    ibrav=0,
    celldm(1)=4.6595,
    nat=2,
    ntyp=1,
    ecutwfc=70.0,
    nbnd = 22,
/
&ELECTRONS
/
ATOMIC_SPECIES
    C 12 C.pbe-rrkjus.UPF
CELL_PARAMETERS
    1.0 0.0 0.0
    -0.5 0.8660254 0.0
    0.0 0.0 30.0
ATOMIC_POSITIONS (crystal)
   C        0   0   0
   C        0.6666666 0.3333333 0
K_POINTS crystal_b
   3
   0 0 0 20
   0.33333 0.33333 0 10
   0.5 0 0 10
\end{Verbatim}
\newpage
\begin{center}
    Приложение В\\
    Файлы конфигурации для CRYSTAL
\end{center}
\addcontentsline{toc}{chapter}{Приложение В Файлы конфигурации для CRYSTAL}
Расчёт электронной плотности
\begin{Verbatim}[fontsize=\footnotesize]
Graphene
SLAB
77
2.46
1
6  0.66666666666 0.33333333333 0.
END
6  2
1 0 3  2.  0.
1 1 3  4.  0.
99 0
END
DFT
EXCHANGE
PBE
CORRELAT
PBE
END
SHRINK
8 8
END
\end{Verbatim}
\vspace{1cm}
Расчёт зонной структуры
\begin{Verbatim}[fontsize=\footnotesize]
NEWK
16 16
1 0
BAND
bands of graphene
2 6 100 3 10 1 0
3 0 0 0 0 0
0 0 0 2 2 0
END
\end{Verbatim}
\newpage
\begin{center}
    Приложение Г\\
    Скрипт для обработки и визуализации данных
\end{center}
\addcontentsline{toc}{chapter}{Приложение Г Скрипт для обработки и визуализации данных}
\begin{Verbatim}[fontsize=\footnotesize]
ab = Import["ab.csv"];
qe = Import["qe.csv"];
cr = Import["cr.csv"];

perm[data_, p_] := Module[{a},
    a = data;
    Do[
        Do[
            If[ Abs[a[[j]] - p[[i]]] < Abs[a[[i]] - p[[i]]],
                {a[[i]], a[[j]]}={a[[j]], a[[i]]}],
            {j, i, Length[p]}],
        {i, Length[p]}];
    a]

bands[data_] := Module[{result},
    result = data[[;;2]];
    Do[
        AppendTo[result,perm[data[[i]], 2 result[[-1]] - result[[-2]]]],
        {i, 3, Length[data]}];
    Transpose[result]]

result = Join[bands[ab],bands[qe],bands[cr]];
Export["bands.png",
        ListLinePlot[result, PlotRange -> All]];
\end{Verbatim}